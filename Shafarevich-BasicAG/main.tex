\documentclass[UTF8]{ctexbook}

%下面开始调用宏包
\usepackage[dvipsnames, svgnames, x11names]{xcolor}% 一般放得靠前 颜色教程: https://www.jianshu.com/p/5aee7c366369
\usepackage{geometry}
\geometry{scale=0.8}
\usepackage{hyperref}
\usepackage{framed}
\usepackage{amsthm}
\usepackage{amsmath}
\usepackage{amssymb}
\usepackage{amsfonts}
\usepackage{enumitem}
\setlist[enumerate,1]{label=(\arabic*)}
\usepackage{bm}
\usepackage{mathrsfs}            %数学花体
\usepackage{graphicx, subfig}%调用图片
%newtheorem的用法:\newtheorem{环境名}{数学内容}[section]
%newtheorem的用法(二):\newtheorem{环境名}[numberedlike]{数学内容}.
%所谓数学内容就是打出来显示的内容,环境名指的是下文中调用该环境的名称.numberedlike是指跟随谁谁谁编号
\usepackage{array}



\title{基础代数几何 \\--投影空间中的代数簇}
\author{Shafarevich}
\date{\today}

\begin{document}
	\maketitle
	\tableofcontents

	\chapter{基础概念}

	\section{平面上的代数曲线}%
	\label{sec:平面上的代数曲线}
	
	\section{仿射空间中的闭子集}%
	\label{sec:仿射空间中的闭子集}
	
	\section{有理函数}%
	\label{sec:有理函数}
	
	\section{拟投影簇}%
	\label{sec:拟投影簇}
	
	\section{拟投影簇的积和映射}%
	\label{sec:拟投影簇的积和映射}
	
	\section{维数}%
	\label{sec:维数}
	

	\chapter{局部性质}

	\chapter{除子和微分形式}

	\chapter{相交数}
\end{document}

